\section{Парадигмы программирования (бонус)}

\subsection{Процедурное программирование}
Обычно код, который пишется для олимпиадного программирования - процедурный.
Это значит, что программа состоит из последовательности инструкций (функций), изменяющих состояние
(изменяет и использует переменные в глобальной области, а не только на аргументы)\\
\textbf{Отличительные черты:}

- \textbf{Императивный стиль}: программа — это последовательность команд.

- \textbf{Функции} как основные блоки кода (не привязаны к объектам).

- \textbf{Быстр} в написании, но неудобен в чтении сторонним разработчиком

- \textbf{Сложен} в поддержке крупных проектов

\subsubsection{Объектно-ориентированное программирование (ООП)}
Думаю, любой начинающий программист задумывался: ``Можно ли создать свой тип данных?'', и ответ - да.
Главная черта ООП - создание своих типов данных

Решение большинства задач через ООП зачастую требует большого времени, чем через процедурное программирование,
зато другой человек сможет понять и использовать его в разы быстрее, поэтому он идеален для создания библиотек.
Например, std::vector, std::string, std::map, std::set и многие другие написаны через классы.
Вообще всё, что не является примитивом (то есть числом, ссылкой или указателем)
В олимпиадном программировании самый подходящий пример для ООП пример - СНМ.

В c++ есть две синтаксические конструкции для создания типов данных: структуры (пришли ещё из языка си) и классы.\\
\textbf{Начнём со структур:}\\
Базовый синтаксис:

\begin{minted}[linenos, mathescape]{cpp}
struct Struct { // где struct - ключевое слово, Struct - имя структуры

};
\end{minted}