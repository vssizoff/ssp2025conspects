\section{Динамическое программирование}

\subsection{Идея динамического программирования}

Идея динамического программирования заключается в том, что если у нас есть задача для какого-то числа $n$, то можно научиться решать её для $n - 1$ и понять, как переходить к следующему числу. 

Можно применять динамику на графах. Например, если мы хотим узнать количество способов, которым можно добраться из одной вершины в другую, то можно применить следующую идею: для стартовой вершины количество способов прийти (из неё же) в неё равно $1$. А для каждой следующей вершины количество способов прийти в неё равно сумме количеств способов прийти в её соседей.

Другой пример --- задача о количестве способов выдать сдачу. Пусть мы имеем монеты номиналами $1$, $2$ и $5$ рублей. Сколькими способами мы можем этими монетами выдать сдачу в размере $8$ рублей?

\begin{center}
\begin{tabular}{| c | c | c | c | c | c | c | c | c |}
    \rowcolor{lightgray}
    \hspace{.1cm} $0$ \hspace{.1cm} & \hspace{.1cm} $1$ \hspace{.1cm} & \hspace{.1cm} $2$ \hspace{.1cm} & \hspace{.1cm} $3$ \hspace{.1cm} & \hspace{.1cm} $4$ \hspace{.1cm} & \hspace{.1cm} $5$ \hspace{.1cm} & \hspace{.1cm} $6$ \hspace{.1cm} & \hspace{.1cm} $7$ \hspace{.1cm} & \hspace{.1cm} $8$ \hspace{.005cm} ${}$\\
    \hline
    $1$ & $0$ & $0$ & $0$ & $0$ & $0$ & $0$ & $0$ & $0$\\
\end{tabular}
\end{center}

Очевидно, количество способов выдать сдачу в размере $0$ рублей, равно $1$ (не берём ни одну монету). В $1$ можно прийти только из $0$, то есть, тоже одним способом. В $2$ можной прийти двумя способами: из $1$ и из $0$. В тройку можем прийти из $1$ и из $2$. Значит, количество способов прийти в $3$ равно сумме количеств способов прийти в $1$ и $2$, то есть, $3$. Затем для~$4$ имеем $3 + 2 = 5$ способов, а для $5$ уже $1 + 3 + 5 = 9$ способов (могли прийти из $0$, $3$ и $4$). Аналогично считаем до конца и получаем, что для $8$ посчитанное нами количество способов равно $44$.

\begin{center}
\begin{tabular}{| c | c | c | c | c | c | c | c | c |}
    \rowcolor{lightgray}
    \hspace{.1cm} $0$ \hspace{.1cm} & \hspace{.1cm} $1$ \hspace{.1cm} & \hspace{.1cm} $2$ \hspace{.1cm} & \hspace{.1cm} $3$ \hspace{.1cm} & \hspace{.1cm} $4$ \hspace{.1cm} & \hspace{.1cm} $5$ \hspace{.1cm} & \hspace{.1cm} $6$ \hspace{.1cm} & \hspace{.1cm} $7$ \hspace{.1cm} & \hspace{.1cm} $8$ \hspace{.005cm} ${}$\\
    \hline
    $1$ & $1$ & $2$ & $3$ & $5$ & $9$ & $15$ & $26$ & $44$\\
\end{tabular}
\end{center}

\subsection{Задача о черепашке}

Пусть имеем поле $n \times m$, на клетках которых растут цветы (в клетке $i, j$ растёт $a_{i, j}$ цветов). Черепашка стартует в клетке $0, 0$ и может идти только на одну клетку вниз или вправо. Она хочет дойти до клетки $n - 1, m - 1$, собрав максимальное количество цветов. Нужно найти это максимальное количество.

Заведём новую таблицу размера $n \times m$, в которой будем для клетки $i, j$ отмечать, какое максимальное количество цветков можно собрать по~пути в неё. Назовём эту таблицу $d$. Заметим, то $d_{k, 0} = \sum\limits_{k = 0}^{k}a_{i, 0}$, $d_{0, k} \hm = \sum\limits_{j = 0}^{k}a_{0, j}$ (в эти клетки мы могли прийти только сверху или слева, причём нам всегда выгодно собирать все цветы с поля). А теперь поймём, как заполнять остальные клетки $d_{i, j}$. Прийти в клетку $(i, j)$ мы могли из клетки $(i - 1, j)$ или $(i, j - 1)$. Мы хотим приходить из той клетки, в которой значение (уже посчитанное из динамики) больше. Тогда для $d_{i, j} = a_{i, j} + \max\{d_{i - 1, j}, d_{i, j - 1}\}$ при $i, j \geqslant 1$. Таким образом, имеем \textit{базу динамики} $d_{0, 0} = 0$ и \textit{формулу перехода}:
\[
    d_{i, j} = 
    \begin{cases}
        d_{i, j} = d_{i - 1, j} + a_{i, j},&\text{если $j = 0$},\\
        d_{i, j} = d_{i, j - 1} + a_{i, j},&\text{если $i = 0$},\\
        d_{i, j} = \max\{d_{i - 1, j}, d_{i, j - 1}\} + a_{i, j},&\text{если $i, j \ne 0$}.
    \end{cases}
\]

\noindent
Для примера рассмотрим такую матрицу $a$:

\begin{center}
\begin{tabular}{| C{.6cm} | C{.6cm} | C{.6cm} | C{.6cm} | C{.6cm} | C{.6cm} | C{.6cm} |}
    \hline
    \cellcolor{green!25}$0$ & $3$ & $0$ & $1$ & $0$ & $11$ & $1$\\
    \hline
    $0$ & $0$ & $2$ & $10$ & $1$ & $2$ & $1$\\
    \hline
    $0$ & $0$ & $0$ & $4$ & $2$ & $0$ & $0$\\
    \hline
    $9$ & $9$ & $3$ & $0$ & $1$ & $4$ & $0$\\
    \hline
    $0$ & $5$ & $0$ & $13$ & $0$ & $2$ & $0$\\
    \hline
    $8$ & $15$ & $0$ & $0$ & $3$ & $0$ & $0$\\
    \hline
\end{tabular}
\end{center}

\noindent
Заполнив её по таким правилам, мы получим следующее:

\begin{center}
\begin{tabular}{| C{.6cm} | C{.6cm} | C{.6cm} | C{.6cm} | C{.6cm} | C{.6cm} | C{.6cm} |}
    \hline
    \cellcolor{green!25}$0$ & $3$ & $3$ & $4$ & $4$ & $15$ & $16$\\
    \hline
    \cellcolor{green!25}$0$ & $3$ & $5$ & $15$ & $16$ & $18$ & $19$\\
    \hline
    \cellcolor{green!25}$0$ & $3$ & $5$ & $19$ & $21$ & $21$ & $21$\\
    \hline
    \cellcolor{green!25}$9$ & \cellcolor{green!25}$18$ & $21$ & $21$ & $22$ & $26$ & $26$\\
    \hline
    $9$ & \cellcolor{green!25}$23$ & $23$ & $36$ & $36$ & $38$ & $38$\\
    \hline
    $17$ & \cellcolor{green!25}$38$ & \cellcolor{green!25}$38$ & \cellcolor{green!25}$38$ & \cellcolor{green!25}$41$ & \cellcolor{green!25}$41$ & \cellcolor{green!25}$41$\\
    \hline
\end{tabular}
\end{center}

Чтобы восстановить путь для какой-либо клетки, достаточно посмотреть, откуда мы в эту клетку пришли. Для этого из верхней и левой клетки выбираем максимальную по записанному значению и идём в неё.

