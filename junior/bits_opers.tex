\section{Битовый бой}

\subsection{Двоичная система счисления}

\begin{definition}[Системы счисления]
    \[
        \overline{a_{n - 1}\ldots a_1 a_0}_x \vcentcolon = \sum\limits_{i = 0}^{n - 1} a_i\cdot x^i
    \]
\end{definition}

В компьютере числа хранятся в двоичной системе счисления, причём хранение отрицательных чисел производится с помощью, так называемого, \textit{обратного кода}. Первый бит отвечает за знак числа (для неотрицательных он равен $0$, для отрицательных $1$). Пусть имеем двоичное число  $A = 1101\underbrace{0\ldots0}_{28\, \text{нулей}}$. Как найти обратное к нему? Чтобы арифметика работала, нужно, чтобы такое число $B$ при сложении с $A$ давало $0$. Этого можно добиться переполнением, то есть, нам нужно получить число $1\underbrace{0\ldots0}_{32\, \text{нуля}}$, что~в типе \texttt{int} является нулём (ведь он хранит только 32 младших бита). Так, нам подойдёт число $B = 0011\underbrace{0\ldots0}_{28\,\text{нулей}}$.

\subsection{Битовые операции}

\begin{definition}
    \begin{enumerate}[nolistsep]
        \item \textit{Конъюнкция} --- побитовое <<и>> ($x\;\&\;y$);
        \item \textit{Дизъюнкция} --- побитовое <<или>> ($x\;\vert\;y$);
        \item \textit{XOR} --- сложение $\bmod\;2$ ($x \wedge y$);
        \item \textit{Инверсия} --- побитовое отрицание ($\sim x$);
        \item \textit{Побитовый сдвиг вправо} --- умножение на $2^k$ ($x\;\texttt{<<}\;k$);
        \item \textit{Побитовый сдвиг влево} --- целочисленное деление на $2^k$ ($x\;\texttt{>>}\;k$).
    \end{enumerate}
\end{definition}

\noindent
\begin{minipage}{.25\textwidth}
$$\def\arraystretch{.8}
\begin{array}{r}
\&
\begin{array}{r}
110100\\

100111\\
\end{array}\\
\hline
\def\arraystretch{.2}
\begin{array}{r}\\
100100
\end{array}
\end{array}
$$
\end{minipage}
\begin{minipage}{.25\textwidth}
$$\def\arraystretch{.8}
\begin{array}{r}
|
\begin{array}{r}
110100\\

100111\\
\end{array}\\
\hline
\def\arraystretch{.2}
\begin{array}{r}\\
110111
\end{array}
\end{array}
$$
\end{minipage}
\begin{minipage}{.25\textwidth}
$$\def\arraystretch{.8}
\begin{array}{r}
\wedge
\begin{array}{r}
110100\\

100111\\
\end{array}\\
\hline
\def\arraystretch{.2}
\begin{array}{r}\\
010011
\end{array}
\end{array}
$$
\end{minipage}
\begin{minipage}{.25\textwidth}
$$\def\arraystretch{.8}
\begin{array}{r}
\begin{array}{r}
{}\\
\sim110100
\end{array}\\
\hline
\def\arraystretch{.2}
\begin{array}{r}\\
001011
\end{array}
\end{array}
$$
\end{minipage}

\problem{Определить значение $i$-го бита числа $x$}

\solution{$x\;\&\;(1\;\texttt{<<}\;i)$}

\problem{Инвертировать $i$-й бит числа $x$}

\solution{$x \wedge (1\;\texttt{<<}\;i)$}

\subsection{Битовые маски}

Заметим, что любое множество (из не более чем 32 элементов) можно задавать двоичной строкой (элементы кодируются $1$ на соответствующих местах). Например, строка $0011001$ задаёт множество $\{0, 3, 4\}$. Заметим, что это даёт нам возможность быстро использовать операции над множествами (им соответствую операции над двоичными числами). Пусть $f$~---~функция, которая по битовой маске строит соответствующее ей множество, $A$ и $B$ --- некоторые множества. Некоторые примеры операций:

$$
    \begin{array}{l}
    A \cap B \leftarrow f^{-1}(A)\; \&\; f^{-1}(B),\\
    A \cup B \leftarrow f^{-1}(A)\; |\; f^{-1}(B),\\
    A\:\Delta\:B \leftarrow f^{-1}(A)\; \wedge\; f^{-1}(B),\\
    \overline{A} \leftarrow\ \sim f^{-1}(A),\\
    A \setminus B \leftarrow f^{-1}(A)\; \&\; (\sim f^{-1}(B)).
\end{array}$$

\subsection{Bitset-ы}

Т.\,к. часто приходится работать с большими множествами (количество элементов в которых больше 32 и даже 64), нам не подходят для хранения обычные числовые типы вроде \texttt{int} и \texttt{long long}. Как раз для этого существует \texttt{bitset<SIZE>} (\texttt{SIZE} --- обязательно константа). Он предоставляет все операции над двоичными числами (единственное условие --- они должны быть одинакового размера).

Помимо увеличенного размера двоичных чисел, \texttt{bitset}-ы нужны, чтобы уменьшать время работы операций, так как в нём биты хранятся не отдельно, а в составе байта (группами по 8 штук), поэтому операции с~\texttt{bitset}-ом работают с очень маленькой константой (как раз в 8 раз меньше, чем при применении битовых операций к обычным числам).

